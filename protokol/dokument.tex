\documentclass{article}

\usepackage[czech]{babel}
\usepackage[utf8]{inputenc}
\usepackage{amssymb,amsmath}
\usepackage{color}

% sazeni zdrojaku
\usepackage{listings}

\definecolor{javared}{rgb}{0.6,0,0} % for strings
\definecolor{javagreen}{rgb}{0.25,0.5,0.35} % comments
\definecolor{javapurple}{rgb}{0.5,0,0.35} % keywords
\definecolor{javadocblue}{rgb}{0.25,0.35,0.75} % javadoc
 
\lstset{language=Java,
basicstyle=\ttfamily,
keywordstyle=\color{javapurple}\bfseries,
stringstyle=\color{javared},
commentstyle=\color{javagreen},
morecomment=[s][\color{javadocblue}]{/**}{*/},
numbers=left,
numberstyle=\tiny\color{black},
stepnumber=1,
numbersep=10pt,
tabsize=4,
showspaces=false,
showstringspaces=false}

\title{Paralelní programování}
\author{Robert Čížek, Lukáš Hlavatý, Peter Smatana}

\begin{document}
\maketitle

\textbf{Zadání projektu}
\newline
Cílem projektu bylo vytvořit program který se snaží hrubou silou rozšifrovat šifrový text.
Šifrovaný text je vytvořený pomocí Vigenèrovy šifry.
\newline

\textbf{Požadavky na softwarové vybavení}
Projekt je vytvořený v progamovacím jazyku Java ve verzi 1.7. Pro vytvoření stuktury projektu,
pro běh a tvorbu \textit{.jar} souborů používáme technologii Maven. Pro běh projektu je třeba
mít nainstalované Java Runtime Environment. Pro práci na projektu je třeba mít Java Development
Kit.
\newline

\textbf{Popis projektu}
\newline
Naše řešení je schopno text šifrovat i dešifrovat. Šifrování se provádí ve třídě
\texttt{Encryptor} v metodě \texttt{Encrypt}. V této metodě iterujeme nad otevřeným textem
a podle klíče jej šifrujeme. O to se stará třída \texttt{Alphabet}, která má metody
\texttt{shiftUp} a \texttt{shiftDown}. Metoda \texttt{shiftUp} šifruje text podle klíče
a metoda \texttt{shiftDown} dešifruje text podle klíče. Ve skutečnosti se jedná o hledání
patřičného znaku ve Vigenèrově čtverci.

Ve zdrojovém kódu č. 1 je vidět jak se pracuje Vigenèrovým čtvercem. Nalezení zašifrovaného znaku
je přičtení ordinální pozice znaku klíče v ASCII tabulce.

\begin{center}
\begin{lstlisting}
public static char shiftUp(char c, char key) {
	int num = (int) c - 97;
	if ((num >= 0) && (num <= 25)) {
		num += (int) key - 97;
		if (num > 25) {
			num -= 26;
		}
		return alphabet.charAt(num);
	} else {
		return c;
	}
}
\end{lstlisting}
\vspace{1mm}
\textit{Zdrojový kód č. 1: Šifrování znaku podle znaku v klíči.}
\end{center}

Pro dešifrování máme třídu \texttt{Decryptor} ve které jsou tři metody kde uplatňujeme
odlišné přístupy na dešifrování. První metoda \texttt{KeyDecrypt} dešifruje text inverzně
stejně jako je implementováno šifrování. Vezme šifrový text, klíč a obdobně jako v ukázce
zdrojového kódu č. 1 dešifruje text. Druhý přistup reprezentuje metoda \texttt{FreqDecrypt},
která provádí frekvenční analýzu. Frekvenční analýza je přístup, jak najít v šifrovém textu
nějakou podobnost s přirozeným jazykem, ve kterém je zpráva napsaná. Poslední přístup je
hledání klíče hrubou silou podle slovníku. Hádáme klíč, klíčem rozšifrováváme slovo a toto
slovo hledáme ve slovníku.
\newline

\textbf{Paralelní vylepšení programu}
\newline
asd asd asd
\newline

\textbf{Data}
\newline
\begin{tabular}{ l | l | l }
  \textbf{název souboru} & počet znaků & počet slov \\
  \hline
  nesmysl.txt & 0 & 0 \\
  pohadka.txt & 0 & 0 \\
  shakespeare.txt 0 & 0 \\
  \hline
\end{tabular}
\newline

\textbf{Měření}
\newline
takhle se delaji v \LaTeX u tabulky:
\begin{tabular}{ l | c | r }
  \hline			
  1 & \textbf{xy} & 3 \\
  \hline
  \hline
  4 & 5 & 6 \\
  7 & 8 & 9 \\
  \hline
\end{tabular}

\textbf{Závěr}
\newline
asd asd asd
\newline

\end{document}










































