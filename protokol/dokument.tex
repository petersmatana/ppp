\documentclass{article}

\usepackage[czech]{babel}
\usepackage[utf8]{inputenc}
\usepackage{amssymb,amsmath}
\usepackage{color}

% sazeni zdrojaku
\usepackage{listings}

\definecolor{javared}{rgb}{0.6,0,0} % for strings
\definecolor{javagreen}{rgb}{0.25,0.5,0.35} % comments
\definecolor{javapurple}{rgb}{0.5,0,0.35} % keywords
\definecolor{javadocblue}{rgb}{0.25,0.35,0.75} % javadoc
 
\lstset{language=Java,
basicstyle=\ttfamily,
keywordstyle=\color{javapurple}\bfseries,
stringstyle=\color{javared},
commentstyle=\color{javagreen},
morecomment=[s][\color{javadocblue}]{/**}{*/},
numbers=left,
numberstyle=\tiny\color{black},
stepnumber=1,
numbersep=10pt,
tabsize=4,
showspaces=false,
showstringspaces=false}

\title{Paralelní programování - semestrální projekt}
\author{Robert Čížek, Lukáš Hlavatý, Peter Smatana}

\begin{document}
\maketitle

\section{Zadání projektu}
Cílem projektu bylo vytvořit program který se snaží hrubou silou rozšifrovat šifrový text.
Šifrovaný text je vytvořený pomocí Vigenèrovy šifry.
\newline

\section{Požadavky na softwarové vybavení}
Projekt je vytvořený v progamovacím jazyku Java ve verzi 1.7. Pro vytvoření stuktury projektu,
pro běh a tvorbu \textit{.jar} souborů používáme technologii Maven. Pro běh projektu je třeba
mít nainstalované Java Runtime Environment. Pro práci na projektu je třeba mít Java Development
Kit.
\newline

\section{Popis projektu}
Naše řešení je schopno text šifrovat i dešifrovat. Šifrování se provádí ve třídě
\texttt{Encryptor} v metodě \texttt{Encrypt}. V této metodě iterujeme nad otevřeným textem
a podle klíče jej šifrujeme. O to se stará třída \texttt{Alphabet}, která má metody
\texttt{shiftUp} a \texttt{shiftDown}. Metoda \texttt{shiftUp} šifruje text podle klíče
a metoda \texttt{shiftDown} dešifruje text podle klíče. Ve skutečnosti se jedná o hledání
patřičného znaku ve Vigenèrově čtverci.

Ve zdrojovém kódu č. 1 je vidět jak se pracuje Vigenèrovým čtvercem. Nalezení zašifrovaného znaku
je přičtení ordinální pozice znaku klíče v ASCII tabulce.

\begin{center}
\begin{lstlisting}
public static char shiftUp(char c, char key) {
	int num = (int) c - 97;
	if ((num >= 0) && (num <= 25)) {
		num += (int) key - 97;
		if (num > 25) {
			num -= 26;
		}
		return alphabet.charAt(num);
	} else {
		return c;
	}
}
\end{lstlisting}
\vspace{1mm}
\textit{Zdrojový kód č. 1: Šifrování znaku podle znaku v klíči.}
\end{center}

Pro dešifrování máme třídu \texttt{Decryptor} ve které jsou tři metody kde uplatňujeme
odlišné přístupy na dešifrování. První metoda \texttt{KeyDecrypt} dešifruje text inverzně
stejně jako je implementováno šifrování. Vezme šifrový text, klíč a obdobně jako v ukázce
zdrojového kódu č. 1 dešifruje text. Druhý přistup reprezentuje metoda \texttt{FreqDecrypt},
která provádí frekvenční analýzu. Frekvenční analýza je přístup, jak najít v šifrovém textu
nějakou podobnost s přirozeným jazykem, ve kterém je zpráva napsaná. Poslední přístup je
hledání klíče hrubou silou podle slovníku. Hádáme klíč, klíčem rozšifrováváme slovo a toto
slovo hledáme ve slovníku.
\newline

\section{Paralelní vylepšení programu}
Paralelizaci jsme prováděli pro dešifrování textu. Šifrování jsme implementovali pro
potřebu získání zašifrovaného textu.

Dešifrování je provedeno rozdělením šifrovaného textu do částí které jsou následně
předány vláknům (viz zdrojový kód č. 2).
Vlákna nezávisle na sobě zpracovávají dané části a výsledky zapisují
do společné proměnné pomocí metod nebo do patřičného pole podle svého ID. Počet
spuštěných vláken je závislý na délce šifrovaného textu.

\begin{center}
\begin{lstlisting}
public static String Encrypt(String text, String key){
	sKey = key;
	StringBuilder result = new StringBuilder();
	String[] batch = Stringer.Split(text);
	Thread[] bank = new Thread[batch.length];
	count = new CountDownLatch(batch.length);
	sBatch = new String[batch.length];
	for(int i = 0; i < batch.length; i++){
		bank[i] = new Thread(new Encryptor(batch[i], i));
		bank[i].start();
	}
	
	try {
		count.await();
	} catch (InterruptedException ex) {
		Logger.getLogger(
			Encryptor.class.getName()).log(Level.SEVERE, null, ex);
	}
	
	for (String s : sBatch) {
		result.append(s);
	}
	return result.toString();
}
\end{lstlisting}
\vspace{1mm}
\textit{Zdrojový kód č. 2: Rozdělení textu a předání vláknům.}
\end{center}

\subsection{Použitý hardware}
Tabulka č. 7 ukazuje, na jakém hardware projekt běžel. Přestože je hardware
různě výkonný, na výsledky to prakticky nemělo vliv. Za příčinu obdobných
výsledků lze především považovat stejný počet procesorových jader, stejně
tak jako absence technologie hyperthreading u starších procesorů značky Intel.
Procesory se tak liší pouze v taktovaní, jež v paralelním zpracování nemá výrazný vliv.
\newline
\begin{center}
	\begin{tabular}{ l | l }
		\textbf{Procesor} & \textbf{Paměť} \\
		\hline
		\hline
		Intel Pentium DualCore E6300 & 4BG DDR2 \\
		\hline
		Intel Core i3-2340UE & 4GB DDR3 \\
		\hline
		AMD Phenom X2 & 4GB DDR2 \\
	\end{tabular}
	\newline
	\textit{Tabulka č. 7: Použitý hardware pro testování}
\end{center}

\section{Práce na projektu}
Na projektu se všichni členové týmu podíleli stejným dílem. Projekt jsme
umístili na github: \texttt{https://github.com/petersmatana/ppp}

\section{Měření}
Používali jsme tři zdroje dat pro šifrování a dešifrování (viz tabulka č. 1).
\begin{center}
	\begin{tabular}{ c | c | c }
	  \textbf{název souboru} & \textbf{počet znaků} & \textbf{počet slov} \\
	  \hline
	  \hline
	  blabol1k.txt & 1 134 & 147 \\
	  \hline
	  blabol10k.txt & 11 224 & 1 495 \\
	  \hline
	  blabol40k.txt & 44844 & 6005 \\
	  \hline
	\end{tabular}
	\newline
	\textit{Tabulka č. 1: Testovací data}
\end{center}


\begin{center}
	\begin{tabular}{ c | c | c | c | c | c | c }
		\textbf{číslo měření} & \multicolumn{3}{c|}{\textbf{lineární zpracování}} & \multicolumn{3}{|c}{\textbf{paralelní zpracování}} \\
		\hline
		& 1k & 10k & 40k & 1k & 10k & 40k \\
		\hline
		\hline
		1 & 2,45 & 3,36 & 9,11 & 2,63 & 2,92 & 2,46 \\
		\hline
		2 & 2,51 & 3,10 & 8,88 & 2,55 & 2,29 & 2,66 \\
		\hline
		3 & 2,61 & 4,07 & 8,84 & 2,54 & 2,50 & 2,56 \\
		\hline
		4 & 3,03 & 3,46 & 9,79 & 2,60 & 2,46 & 2,69 \\
		\hline
		5 & 2,42 & 3,01 & 8,55 & 2,44 & 2,66 & 2,75 \\
		\hline
		6 & 2,46 & 3,27 & 8,48 & 2,41 & 2,43 & 2,63 \\
		\hline
		7 & 2,29 & 3,07 & 9,06 & 2,49 & 2,57 & 2,48 \\
		\hline
		8 & 2,87 & 3,65 & 8,58 & 2,52 & 2,53 & 2,66 \\
		\hline
		9 & 2,78 & 2,86 & 8,78 & 2,53 & 2,62 & 2,58 \\
		\hline
		10 & 2,50 & 3,43 & 8,82 & 2,46 & 2,43 & 2,71 \\
		\hline
		průměr & 2,59 & 3,33 & 8,89 & 2,52 & 2,54 & 2,62 \\
	\end{tabular}
	\newline
	\textit{Tabulka č. 2: KeyDecrypt}
\end{center}

\begin{center}
	\begin{tabular}{ c | c | c | c | c | c | c }
		\textbf{číslo měření} & \multicolumn{3}{c|}{\textbf{lineární zpracování}} & \multicolumn{3}{|c}{\textbf{paralelní zpracování}} \\
		\hline
		& 1k & 10k & 40k & 1k & 10k & 40k \\
		\hline
		\hline
		1 & 2,95 & 3,04 & 6,29 & 2,81 & 2,93 & 2,72 \\
		\hline
		2 & 2,67 & 2,92 & 6,77 & 2,39 & 2,97 & 2,81 \\
		\hline
		3 & 2,79 & 2,82 & 6,16 & 2,55 & 2,57 & 2,79 \\
		\hline
		4 & 2,42 & 2,87 & 6,04 & 2,94 & 2,64 & 2,84 \\
		\hline
		5 & 2,41 & 2,75 & 6,04 & 2,42 & 2,95 & 2,68 \\
		\hline
		6 & 2,64 & 2,64 & 5,75 & 2,85 & 2,76 & 2,98 \\
		\hline
		7 & 2,44 & 2,88 & 5,85 & 2,97 & 2,65 & 2,90 \\
		\hline
		8 & 2,35 & 3,21 & 5,91 & 2,36 & 2,86 & 2,73 \\
		\hline
		9 & 2,50 & 2,91 & 5,43 & 2,47 & 2,75 & 2,95 \\
		\hline
		10 & 3,16 & 2,75 & 5,65 & 2,69 & 2,81 & 2,74 \\
		\hline
		průměr & 2,63 & 2,88 & 5,99 & 2,64 & 2,79 & 2,81 \\
	\end{tabular}
	\newline
	\textit{Tabulka č. 3: Encrypt}
\end{center}

\begin{center}
	\begin{tabular}{ c | c | c | c | c | c | c }
		\textbf{číslo měření} & \multicolumn{3}{c|}{\textbf{lineární zpracování}} & \multicolumn{3}{|c}{\textbf{paralelní zpracování}} \\
		\hline
		& 1k & 10k & 40k & 1k & 10k & 40k \\
		\hline
		\hline
		1 & 3,38 & 21,05 & 283,92 & 3,11 & 19,62 & 254,40 \\
		\hline
		2 & 2,96 & 20,66 & 291,43 & 3,37 & 19,43 & 255,37 \\
		\hline
		3 & 3,04 & 20,30 & 281,26 & 2,74 & 18,67 & 253,65 \\
		\hline
		4 & 2,82 & 21,06 & 282,64 & 3,16 & 19,85 & 254,19 \\
		\hline
		5 & 3,23 & 20,46 & 281,37 & 2,85 & 18,67 & 253,87 \\
		\hline
		6 & 3,47 & 20,66 & 281,95 & 3,18 & 18,73 & 255,02 \\
		\hline
		7 & 2,96 & 21,09 & 284,34 & 3,06 & 18,45 & 254,41 \\
		\hline
		8 & 3,02 & 20,70 & 280,14 & 2,95 & 19,06 & 254,34 \\
		\hline
		9 & 3,31 & 20,79 & 281,35 & 2,86 & 18,63 & 254,41 \\
		\hline
		10 & 2,81 & 20,15 & 281,48 & 3,11 & 18,71 & 255,17 \\
		\hline
		průměr & 3,10 & 20,69 & 282,99 & 3,04 & 18,98 & 254,48 \\
	\end{tabular}
	\newline
	\textit{Tabulka č. 4: FreqDecrypt}
\end{center}

\begin{center}
	\begin{tabular}{ c | c | c | c }
		\multicolumn{4}{c}{\textbf{overhead}} \\
		\hline
		číslo měření & 1k & 10k & 40k \\
		\hline
		\hline
		1 & 2,66 & 2,50 & 2,58 \\
		\hline
		2 & 2,43 & 2,75 & 2,37 \\
		\hline
		3 & 2,47 & 2,45 & 2,76 \\
		\hline
		4 & 2,47 & 2,60 & 2,34 \\
		\hline
		5 & 2,42 & 2,85 & 2,50 \\
		\hline
		6 & 2,50 & 2,60 & 2,63 \\
		\hline
		7 & 3,14 & 2,40 & 2,33 \\
		\hline
		8 & 2,49 & 2,66 & 2,76 \\
		\hline
		9 & 2,41 & 2,46 & 2,29 \\
		\hline
		10 & 2,35 & 2,54 & 2,58 \\
		\hline
		průměr & 2,54 & 2,58 & 2,51 \\
	\end{tabular}
	\newline
	\textit{Tabulka č. 5: overhead}
\end{center}

\begin{center}
	\begin{tabular}{ c | c | c | c }
		 & Enc & KD & FD \\
		\hline
		\hline
		A 1k & 0,10 & -0,04 & 0,47 \\
		\hline
		A 10k & 0,30 & 0,45 & 17,81 \\
		\hline
		A 40k & 3,48 & 2,90 & 277,00 \\
		\hline
		T 1k & 0,11 & -0,13 & 0,40 \\
		\hline
		T 10k & 0,21 & -0,25 & 16,19 \\
		\hline
		T 40k & 0,30 & -0,20 & 251,67 \\
		\hline
		A-T 1k & -0,01 & 0,08 & 0,07 \\
		\hline
		A-T 10k & 0,09 & 0,70 & 1,62 \\
		\hline
		A-T 40k & 3,18 & 3,09 & 25,33 \\
	\end{tabular}
	\newline
	\textit{Tabulka č. 6: výsledné časy běhu}
\end{center}

\section{Závěr}
Z výsledků v tabulce č. 6 vyplývá, že časová náročnost algoritmu metody Encrypt a Keydecrypt je téměř totožná, ale u algoritmu metody FreqDecrypt je vidět, že její složitost je vyšší a tedy paralelizací ziskává největší časovou výhodu. Hlavní příčinou je paralelizace zpracování samotných částí textu - efekt zrychlení zpracování je zde závislý na velikosti souboru dat. Čím větší vstupní soubor bude, tím lepších výsledků aplikace v paralelním módu dosáhne.
Alternativní možností paralelizace by pak mohlo být dešifrování (hledání) samotného šifrovacího klíče. V principu by se rozdělilo samotné hledání, včetně slovníkových operací na více vláken a tak by se zefektivnil proces opakovaného zpracování velkého množství kombinací. Takové zpracování by pak nebylo závislé na velikosti vstupního souboru, ale na použitém slovníku či množství možných kombinací šifrovacího klíče.

\end{document}
